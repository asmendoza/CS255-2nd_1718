\section{Introduction}
Communication using mobile devices had made significant improvements on the way people communicate with each other. Given the different mobile operating systems that allow mobility - iOS and Android - it introduced countably many mobile applications that implement better and innovative communication other than the built-in call and SMS phone functionality.

Limitations on these mobile communication applications is that they all rely on the network availability - signal strength, data and mobile plan subscription, and hardware specifications. This is evident when there are natural calamities like typhoons, earthquake, volcanic eruptions, etc., from which there is poor or no network service due to damages in cell sites.

In this paper, the researchers will develop a mobile application that will enable text messages to be sent over mobile ad-hoc network or mobile hotspot, with end-to-end encryption, independent of the network signal or working WiFi connection. The mobile application will be based on the encryption strategy applied in Signal Private Messenger mobile application, and Barnacle mobile application for creating mobile ad-hoc networks.

The researchers then will examine the effectivity and efficiency of the mobile application over accuracy of the delivery of data (messages) and the speed of implementation of the features. This will enable them to provide an in-depth analysis on how will communication over these two network topologies affect the performance of the mobile application.

\subsection{Objectives and Research Questions}
This paper aims to achieve the following objectives, with the corresponding questions to answer by the end of the study:
\begin{enumerate}
	\item Create a mobile application that will simulate and implement an encrypted end-to-end messaging service over the mobile ad-hoc network and mobile hotspot.
	\begin{enumerate}
		\item What mobile platform will be used in this research?
		\item What encryption types will be used for end-to-end messaging between devices?
		\item How will ad-hoc network be created in mobile?
		\item How will mobile hotspot be created in mobile?
	\end{enumerate}
	\item  
	\item Analyze and evaluate the performance and implementation of the mobile application over the two network connections.
	\begin{enumerate}
		\item Which network is easier to implement on mobile?
		\item Which network provides faster sending of messages between devices, given that both are already in the network?
		\item What are the limitations of each network type that needs to be addressed in future releases?
	\end{enumerate}
\end{list}
 
\subsection{Significance of the Study}
This research was initiated with the need of communication in times of disaster or calamity when network services are unavailable. This will address the need for faster response during these situations through communication.

\subsection{Scope and Limitation}
This research will focus on creating mobile application that will simulate and implement an encrypted end-to-end messaging service over mobile ad-hoc network or mobile hotspot on the following situations:

\begin{enumerate}
	\item Android will be the platform of choice in creating the mobile application needed. Though iOS platform is also into consideration, it will be included in future releases, or a recommendation for future enhancements.
	\item The research will focus on ad-hoc network and mobile hotspot. 
	\item Performance evaluation and analysis will be between the two network types in terms of speed, efficiency, and accuracy.
\end{enumerate}

\section{Review of Related Literature}
Typically, the body of a paper is organized into a hierarchical
structure, with numbered or unnumbered headings for sections,
subsections, sub-subsections, and even smaller sections.  The command
\texttt{{\char'134}section} that precedes this paragraph is part of
such a hierarchy.\footnote{This is a footnote.} \LaTeX\ handles the
numbering and placement of these headings for you, when you use the
appropriate heading commands around the titles of the headings.  If
you want a sub-subsection or smaller part to be unnumbered in your
output, simply append an asterisk to the command name.  Examples of
both numbered and unnumbered headings will appear throughout the
balance of this sample document.

Because the entire article is contained in the \textbf{document}
environment, you can indicate the start of a new paragraph with a
blank line in your input file; that is why this sentence forms a
separate paragraph.

\section{Methodology}
This paragraph will end the body of this sample document.
Remember that you might still have Acknowledgments or
Appendices; brief samples of these
follow.  There is still the Bibliography to deal with; and
we will make a disclaimer about that here: with the exception
of the reference to the \LaTeX\ book, the citations in
this paper are to articles which have nothing to
do with the present subject and are used as
examples only.
%\end{document}  % This is where a 'short' article might terminate

\appendix
%Appendix A
\section{Headings in Appendices}

\section{More Help for the Hardy}

Of course, reading the source code is always useful.  The file
\path{acmart.pdf} contains both the user guide and the commented
code.

\begin{acks}
  The authors would like to thank Dr. Yuhua Li for providing the
  MATLAB code of the \textit{BEPS} method.

  The authors would also like to thank the anonymous referees for
  their valuable comments and helpful suggestions. The work is
  supported by the \grantsponsor{GS501100001809}{National Natural
    Science Foundation of
    China}{http://dx.doi.org/10.13039/501100001809} under Grant
  No.:~\grantnum{GS501100001809}{61273304}
  and~\grantnum[http://www.nnsf.cn/youngscientists]{GS501100001809}{Young
    Scientists' Support Program}.

\end{acks}
